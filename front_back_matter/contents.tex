% Table of Contents - List of Tables/Figures/Listings and Acronyms

\refstepcounter{dummy}

\pdfbookmark[1]{\contentsname}{tableofcontents} % Bookmark name visible in a PDF viewer

\setcounter{tocdepth}{2} % Depth of sections to include in the table of contents - currently up to subsections

\setcounter{secnumdepth}{3} % Depth of sections to number in the text itself - currently up to subsubsections

\manualmark
\markboth{\spacedlowsmallcaps{\contentsname}}{\spacedlowsmallcaps{\contentsname}}
\tableofcontents 
\automark[section]{chapter}
\renewcommand{\chaptermark}[1]{\markboth{\spacedlowsmallcaps{#1}}{\spacedlowsmallcaps{#1}}}
\renewcommand{\sectionmark}[1]{\markright{\thesection\enspace\spacedlowsmallcaps{#1}}}

\clearpage

\begingroup 
\let\clearpage\relax
\let\cleardoublepage\relax
\let\cleardoublepage\relax

%----------------------------------------------------------------------------------------
%	List of Figures
%----------------------------------------------------------------------------------------

\refstepcounter{dummy}

%\pdfbookmark[1]{\listfigurename}{lof} % Bookmark name visible in a PDF viewer

\listoffigures

\vspace*{8ex}
\newpage

%----------------------------------------------------------------------------------------
%	List of Tables
%----------------------------------------------------------------------------------------

\refstepcounter{dummy}
\listoftables

\vspace*{8ex}
\newpage
    
%-------------------------------------
%	List of Listings
%-------------------------------------


%\refstepcounter{dummy}
\renewcommand\listoflistingscaption{List of source codes}
\listoflistings
\vspace*{8ex}
\newpage

%\addcontentsline{toc}{chapter}{\lstlistlistingname} % Uncomment if you would like the list of listings to appear in the table of contents

%\pdfbookmark[1]{\lstlistlistingname}{lol} % Bookmark name visible in a PDF viewer
%
%\lstlistoflistings 
%
%\vspace*{8ex}
%\newpage
       
%----------------------------------------------------------------------------------------
%	Acronyms
%----------------------------------------------------------------------------------------
\refstepcounter{dummy}
\markboth{\spacedlowsmallcaps{Acronyms}}{\spacedlowsmallcaps{Acronyms}}
\chapter*{Acronyms}

\begin{acronym}[Acronyms]
\acro{OSM}{OpenStreetMap, free map}
\acro{ETL}{Extract, Transform and Load}
\acro{RUP}{Rational Unified Process}
\acro{GIS}{Geographic Information System}
\acro{GDAL}{Geospatial Data Abstraction Library}
\acro{WMS}{Web Map Service}
\acro{DRY}{Don't Repeat Yourself}
\acro{CI}{Continuous Integration}
\acro{CDN}{Content Delivery Network}


\end{acronym}  

\newpage

%----------------------------------------------------------------------------------------
%	Glossary
%----------------------------------------------------------------------------------------
\refstepcounter{dummy}
\markboth{\spacedlowsmallcaps{Glossary}}{\spacedlowsmallcaps{Glossary}}
\chapter*{Glossary}

\begin{acronym}[Glossary]
\acro{Vector Tiles}{Packets of geographic data, packaged into pre-defined roughly-square shaped "tiles" for transfer over the web}
\acro{Data Style}{Description of feature classes such as landuse, water or roads}
\acro{Visual Style}{Definition of style rules for a specific schema, which is defined in the data style}
\acro{Feature Class}{Group of features with the same geometry type and attributes}
\acro{Layer}{Mapbox definition of a feature class}
\acro{Mapbox Streets}{Name of Mapbox's vector tile source}
\acro{MBTiles}{File format for storing map tiles in a single file}
\acro{GeoJSON}{File format for encoding a variety of geographic data structures}
\acro{Mapnik XML}{Stylesheet for the mapnik rendering engine}
\acro{CartoCSS}{Mapbox propritary cartographic styling language}
\acro{Mapbox GL}{Clientside rendering engine}
\acro{Web GL}{Javascript API for the graphics library in browsers}
\acro{Mapbox Studio Classic}{Client application to design custom maps}
\acro{OSM Bright}{Mapbox visual style}
\acro{Docker}{Operation system level virtualization on Linux}
\acro{Kitematic}{Client application for controlling docker containers}
\acro{Natural Earth}{Public map dataset}
\acro{OSM Planet}{All OpenStreetMap data in one file}






\end{acronym}  

\endgroup