%%%%%%%%%%%%%%%%%%%%%%%%%%%%%%%%%%%%%%%%%
% Thesis Configuration File
%
% The main lines to change in this file are in the DOCUMENT VARIABLES
% section, the rest of the file is for advanced configuration.
%
%%%%%%%%%%%%%%%%%%%%%%%%%%%%%%%%%%%%%%%%%

%----------------------------------------------------------------------------------------
%	DOCUMENT VARIABLES
%	Fill in the lines below to enter your information into the thesis template
%	Each of the commands can be cited anywhere in the thesis
%----------------------------------------------------------------------------------------

% Remove drafting to get rid of the '[ Date - classicthesis version 4.0 ]' text at the bottom of every page
\PassOptionsToPackage{eulerchapternumbers,listings,pdfspacing,subfig,beramono,eulermath,parts,dottedtoc,tocaligned}{classicthesis}
% Available options: drafting parts nochapters linedheaders eulerchapternumbers beramono eulermath pdfspacing minionprospacing tocaligned dottedtoc manychapters listings floatperchapter subfig
% Adding 'dottedtoc' will make page numbers in the table of contents flushed right with dots leading to them

\newcommand{\myTitle}{Updatable Vector Tiles from OpenStreetMap\xspace}
\newcommand{\mySubtitle}{Free Vector Tiles from OpenStreetMap data\xspace}
\newcommand{\myThesis}{Bachelor Thesis\xspace}
\newcommand{\myName}{Lukas Martinelli, Manuel Roth\xspace}
\newcommand{\myProf}{Prof. Stefan Keller\xspace}
\newcommand{\myFaculty}{Geometa Lab\xspace}
\newcommand{\myDepartment}{Department of Computer Science\xspace}
\newcommand{\myUni}{HSR University of Applied Science Rapperswil\xspace}
\newcommand{\myLocation}{Rapperswil\xspace}
\newcommand{\myTime}{Spring 2016\xspace}
\newcommand{\myVersion}{version 1.0\xspace}
\newcommand{\myLicense}{CC BY-SA 3.0 Unported\xspace}
\newcommand{\myKeywords}{OpenStreetMap, Mapbox, Vector Tiles, PostGIS, SQL, Cartography, Docker}

%----------------------------------------------------------------------------------------
%	USEFUL COMMANDS
%----------------------------------------------------------------------------------------

\newcommand{\ie}{i.\,e.}
\newcommand{\Ie}{I.\,e.}
\newcommand{\eg}{e.\,g.}
\newcommand{\Eg}{E.\,g.} 

\newcommand{\osm}{\emph{OpenStreetMap}}
\newcommand{\osmvt}{\emph{OSM2VectorTiles}}

\newcounter{dummy} % Necessary for correct hyperlinks (to index, bib, etc.)
\providecommand{\mLyX}{L\kern-.1667em\lower.25em\hbox{Y}\kern-.125emX\@}

%-----------------------------------------------
%	Include pdf
%-----------------------------------------------
\usepackage{pdfpages}

%-----------------------------------------------
%	Support figure inside minipage
%-----------------------------------------------
\usepackage{float}


%-----------------------------------------------
%	Remove indent after figures and tables
%-----------------------------------------------
\setlength{\parindent}{0pt}

%-----------------------------------------------
%	PACKAGES
%-----------------------------------------------

\usepackage{lipsum} % Used for inserting dummy 'Lorem ipsum' text into the template

%------------------------------------------------
 
\PassOptionsToPackage{utf8}{inputenc}
\usepackage{inputenc}
 
%------------------------------------------------

\PassOptionsToPackage{american}{babel}
\usepackage{babel}

%------------------------------------------------			

\PassOptionsToPackage{square,numbers}{natbib}
\usepackage{natbib}
 
%------------------------------------------------

\PassOptionsToPackage{fleqn}{amsmath} % Math environments and more by the AMS 
\usepackage{amsmath}
 
%------------------------------------------------

\PassOptionsToPackage{T1}{fontenc}
\usepackage{fontenc}

%------------------------------------------------

\usepackage{xspace} % To get the spacing after macros right

%------------------------------------------------

\usepackage{mparhack} % To get marginpar right

%------------------------------------------------

\usepackage{fixltx2e} % Fixes some LaTeX stuff 

%------------------------------------------------

\PassOptionsToPackage{smaller}{acronym} % Include printonlyused in the first bracket to only show acronyms used in the text
\usepackage{acronym} % nice macros for handling all acronyms in the thesis
%------------------------------------------------

%\renewcommand*{\acsfont}[1]{\textssc{#1}} % For MinionPro

%------------------------------------------------
\PassOptionsToPackage{pdftex}{graphicx}
\usepackage{graphicx} 
\usepackage{subfig}

%------------------------------------------------
\usepackage{pgf}

%------------------------------------------------
\usepackage{wrapfig}
\usepackage{placeins}


%------------------------------------------------
\usepackage{pdfpages}

%------------------------------------------------
\usepackage{siunitx}

%------------------------------------------------
\usepackage{tcolorbox}

%------------------------------------------------
%	FLOATS: TABLES, FIGURES AND CAPTIONS SETUP
%------------------------------------------------

\usepackage{tabularx} % Better tables
\setlength{\extrarowheight}{3pt} % Increase table row height
\newcommand{\tableheadline}[1]{\multicolumn{1}{c}{\spacedlowsmallcaps{#1}}}
\newcommand{\myfloatalign}{\centering} % To be used with each float for alignment
\usepackage{caption}
\captionsetup{format=hang,font=small}
\usepackage{subfig}  

%------------------------------------------------
%	CODE LISTINGS SETUP
%------------------------------------------------
\usepackage{minted} % Syntax highlighting

\usemintedstyle{friendly}

\newminted{python}{framesep=2mm,samepage=true,fontsize=\footnotesize}

\newminted{yaml}{framesep=2mm,samepage=true,fontsize=\footnotesize}

\newminted{sql}{framesep=2mm,samepage=true,fontsize=\footnotesize}

\newminted{plpgsql}{framesep=2mm,samepage=true,fontsize=\footnotesize}

\newminted{c}{framesep=2mm,samepage=true,fontsize=\footnotesize}

\newminted{json}{framesep=2mm,samepage=true,fontsize=\footnotesize}

\newminted{javascript}{framesep=2mm,samepage=true,fontsize=\footnotesize}

\newminted{bash}{framesep=2mm,samepage=true,fontsize=\footnotesize}

\newminted{html}{framesep=2mm,samepage=true,fontsize=\footnotesize}

\newminted{xml}{framesep=2mm,samepage=true,fontsize=\footnotesize}




%\usepackage{listings} 
%\lstset{emph={trueIndex,root},emphstyle=\color{BlueViolet}}%\underbar} % for special keywords
%\lstset{language=Python, % Specify the language for listings here
%keywordstyle=\color{RoyalBlue}, % Add \bfseries for bold
%basicstyle=\small\ttfamily, % Makes listings a smaller font size and a different font
%%identifierstyle=\color{NavyBlue}, % Color of text inside brackets
%commentstyle=\color{Green}\ttfamily, % Color of comments
%stringstyle=\rmfamily, % Font type to use for strings
%numbers=left, % Change left to none to remove line numbers
%numberstyle=\scriptsize, % Font size of the line numbers
%stepnumber=5, % Increment of line numbers
%numbersep=8pt, % Distance of line numbers from code listing
%showstringspaces=false, % Sets whether spaces in strings should appear underlined
%breaklines=true, % Force the code to stay in the confines of the listing box
%%frameround=ftff, % Uncomment for rounded frame
%frame=single, % Frame border - none/leftline/topline/bottomline/lines/single/shadowbox/L
%belowcaptionskip=.75\baselineskip % Space after the "Listing #: Desciption" text and the listing box
%}

%----------------------------------------------------------------------------------------
%	HYPERREFERENCES
%----------------------------------------------------------------------------------------
\PassOptionsToPackage{pdftex,hyperfootnotes=false,pdfpagelabels}{hyperref}
\usepackage{hyperref}  % backref linktocpage pagebackref
\pdfcompresslevel=9
\pdfadjustspacing=1
\hypersetup{
% Uncomment the line below to remove all links (to references, figures, tables, etc)
%draft, 
colorlinks=true, linktocpage=true, pdfstartpage=1, pdfstartview=FitV,
% Uncomment the line below if you want to have black links (e.g. for printing black and white)
%colorlinks=false, linktocpage=false, pdfborder={0 0 0}, pdfstartpage=1, pdfstartview=FitV, 
breaklinks=true, pdfpagemode=UseNone, pageanchor=true, pdfpagemode=UseOutlines,
plainpages=false, bookmarksnumbered, bookmarksopen=true, bookmarksopenlevel=1,
hypertexnames=true, pdfhighlight=/O, urlcolor=OsmGreen, linkcolor=RoyalBlue, citecolor=OsmGreen,
%------------------------------------------------
% PDF file meta-information
pdftitle={\myTitle},
pdfauthor={\textcopyright\ \myName, \myUni, \myFaculty},
pdfsubject={\mySubtitle},
pdfkeywords={\myKeywords},
pdfcreator={pdfLaTeX},
pdfproducer={LaTeX with hyperref and classicthesis}
%------------------------------------------------
}   

%----------------------------------------------------------------------------------------
%	BACKREFERENCES
%----------------------------------------------------------------------------------------

\usepackage{ifthen} % Allows the user of the \ifthenelse command
\newboolean{enable-backrefs} % Variable to enable backrefs in the bibliography
\setboolean{enable-backrefs}{false} % Variable value: true or false

\newcommand{\backrefnotcitedstring}{\relax} % (Not cited.)
\newcommand{\backrefcitedsinglestring}[1]{(Cited on page~#1.)}
\newcommand{\backrefcitedmultistring}[1]{(Cited on pages~#1.)}
\ifthenelse{\boolean{enable-backrefs}} % If backrefs were enabled
{
\PassOptionsToPackage{hyperpageref}{backref}
\usepackage{backref} % to be loaded after hyperref package 
\renewcommand{\backreftwosep}{ and~} % separate 2 pages
\renewcommand{\backreflastsep}{, and~} % separate last of longer list
\renewcommand*{\backref}[1]{}  % disable standard
\renewcommand*{\backrefalt}[4]{% detailed backref
\ifcase #1 
\backrefnotcitedstring
\or
\backrefcitedsinglestring{#2}
\else
\backrefcitedmultistring{#2}
\fi}
}{\relax} 

%-------------------------------------
%	AUTOREFERENCES SETUP
%	Redefines how references in text are prefaced for different 
%	languages (e.g. "Section 1.2" or "section 1.2")
%-------------------------------------

\makeatletter
\@ifpackageloaded{babel}
{
\addto\extrasamerican{
\renewcommand*{\figureautorefname}{Figure}
\renewcommand*{\tableautorefname}{Table}
\renewcommand*{\partautorefname}{Part}
\renewcommand*{\chapterautorefname}{Chapter}
\renewcommand*{\sectionautorefname}{Section}
\renewcommand*{\subsectionautorefname}{Section}
\renewcommand*{\subsubsectionautorefname}{Section}
}
\addto\extrasngerman{
\renewcommand*{\paragraphautorefname}{Absatz}
\renewcommand*{\subparagraphautorefname}{Unterabsatz}
\renewcommand*{\footnoteautorefname}{Fu\"snote}
\renewcommand*{\FancyVerbLineautorefname}{Zeile}
\renewcommand*{\theoremautorefname}{Theorem}
\renewcommand*{\appendixautorefname}{Anhang}
\renewcommand*{\equationautorefname}{Gleichung}
\renewcommand*{\itemautorefname}{Punkt}
}
\providecommand{\subfigureautorefname}{\figureautorefname} % Fix to getting autorefs for subfigures right
}{\relax}
\makeatother

%-------------------------------------

\usepackage{classicthesis} 

%-------------------------------------
%	MARGINS
%-------------------------------------
\usepackage[left=2.5cm,right=2.5cm,top=2.5cm,bottom=3cm]{geometry}

%-------------------------------------
%	CHANGING TEXT AREA 
%-------------------------------------

%\linespread{1.05} % a bit more for Palatino
%\areaset[current]{312pt}{761pt} % 686 (factor 2.2) + 33 head + 42 head \the\footskip
\setlength{\marginparwidth}{0pt}
\setlength{\marginparsep}{0pt}


%-------------------------------------
%	USING DIFFERENT FONTS
%-------------------------------------

%\usepackage[oldstylenums]{kpfonts} % oldstyle notextcomp
%\usepackage[osf]{libertine}
%\usepackage{hfoldsty} % Computer Modern with osf
%\usepackage[light,condensed,math]{iwona}
%\renewcommand{\sfdefault}{iwona}
%\usepackage{lmodern} % <-- no osf support :-(
%\usepackage[urw-garamond]{mathdesign} <-- no osf support :-(
