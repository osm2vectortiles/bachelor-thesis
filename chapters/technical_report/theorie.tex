\chapter{Theory}

This chapter describes the underlying concepts which are used throughout the thesis. 

\section{XYZ Coordinates}\label{part1_xyz_coordinates}

To reduce the amount of data which has to be loaded, vector tiles make use of slicing the data into tiles. One tile contains the geometry data which is visible in 512 x 512 pixels of the screen. The file structure is organized in a 3-dimensional coordinate system (z/x/y), where z represents the zoom level and x and y the axes. This schema is known as XYZ coordinate schema. OSM2VectorTiles provides tiles down to zoom level 14, which results in 268'435'456 ($2^{2*14}$) tiles.

\section{Vector Tiles}\label{part1_vector_tiles}

Vector tiles contain all the geometry and meta data needed for a specific tile. This makes them more flexible than serving raster tiles, because a different style can be applied on the fly. For example based on the language preference of the browser language-specific city labels can be shown.\\
The data inside of vector tiles is structured into layers and features. A vector tile can have multiple layers such as roads, landuse or water. Each layer consist of one or multiple features. A feature contains a geometry field and meta data such as label translations. The geometry field can be of type point, linestring or polygon. 

\subsection{Mapbox Vector Tile Specification}\label{part1_vector_tile_specification}

The Mapbox Vector Tile Specification defines how to encode tiled vector data. More details about the encoding and internal structure of vector tiles can be found in the specification[https://github.com/mapbox/vector-tile-spec].

\section{Map Styling}\label{part1_map_styling}

Additionally to the vector tiles the map style is needed. Vector tiles define the geometry data and associated meta data, whereas the map style defines how this data should be visually styled.

\subsection{Mapbox GL Style}

The Mapbox GL Style is a JSON object which defines what data to draw, the order to draw it in, and how to style the data when drawing it. [https://www.mapbox.com/mapbox-gl-style-spec/]

The example below defines that the layer called "water" should have a blue fill-color.

\begin{jsoncode}
"layers": [
  {
    "id": "water",
    "source": "mapbox-streets",
    "source-layer": "water",
    "type": "fill",
    "paint": {
      "fill-color": "#1b4dd6"
    }
  }
]
\end{jsoncode}


\section{TM2Source Project}\label{par1_tm2source_project}

The TM2Source Project defines the different layers in the vector tiles. This definition is the handed over to Mapnik which generates the vector tiles based on this information.\\
The listing below shows a typical layer definition in the source project. The data source of this layer is a postgis database. The where clause in the SQL query filters the data to only return data where the geometry is inside the bounding box of the current tile (geometry && !bbox!). And secondly this layer should only be contained in tiles on zoom level 14 (z(!scale_denominator!) = 14).

\begin{yamlcode}
- id: barrier_line
    Datasource: 
      dbname: osm
      extent: -20037508.34,-20037508.34,20037508.34,20037508.34
      geometry_field: ''
      geometry_table: ''
      host: db
      key_field: osm_id
      key_field_as_attribute: false
      max_size: 512
      password: osm
      port: 5432
      srid: ''
      table: |-
        (
          SELECT osm_ids2mbid(osm_id, is_polygon(geometry)) AS osm_id, geometry, barrier_line_class(type) AS class
          FROM barrier_line_z14
          WHERE geometry && !bbox!
            AND z(!scale_denominator!) = 14
        ) AS data
      type: postgis
      user: osm
    description: ''
    fields: 
      class: "One of: fence, hedge, cliff, gate"
    properties: 
      "buffer-size": 4
\end{yamlcode}
