\chapter{Results and Future}\label{part1_results_and_future}

Solutions for all problems defined in \autoref{goals} have been found and implemented.

\section{Results}\label{part1_results}

\begin{itemize}
    \item The quality of the vector tiles has been significantly improved and compatibility with Mapbox Streets v7 has been reached.
    \item An open source workflow to scale rendering of vector tiles with global coverage has been created.
    \item A process for updating vector tiles based on \osm{} Diff files has been implemented.
    \item The project website has been significantly improved based on user feedback.
    \item Vector tiles for the entire planet, 219 country and 692 city extracts are provided as download on the project website.
    \item Detailed tutorials and a custom tile server to make getting started as easy as possible were created.
\end{itemize}

\section{Future}\label{part1_future}

The interest and involvement of many people showed that this project provides great value to the FOSS community. In order for it to be valuable in the future several challenges need to be overcome:

\begin{itemize}
    \item An infrastructure provider needs to found in order to be able to run the update process on a regular basis.
    \item More people need to know about this project
    \item During the project, Mapbox released an update to their vector tile specification (version 2.0). The new standard needs to be supported, in order to properly run with their tools in the future.
\end{itemize}

For some of the challenges possible solutions are already planned in the coming months:

\begin{itemize}
    \item The institute for software at HSR owns a server which could possibly be used for the update process, but this is not yet confirmed.
    \item Lukas Martinelli and Manuel Roth will give a talk about \osmvt{} at the FOSSGIS conference in Salzburg and FOSS4G conference in Bonn. This will help to reach the broader FOSS community and further enhance the interest in this project.
    \item In order to support the Mapbox vector tile 2.0 specification the entire planet needs to be rerendered. If time and resources allow the project maintainers intend to do this in the future.
\end{itemize}


