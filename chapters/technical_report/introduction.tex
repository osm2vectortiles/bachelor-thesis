\chapter{Introduction}

%---------------------------------------


%---------------------------------------

\section{Vision}\label{part1_vision}

Vector tiles are the future of web and mobile mapping. The next generation of maps is only possible
with free and open vector tile sources. The goal of the bachelor thesis is to push mapping forward by providing
vector tiles that are produced using an open process, completely free of charge and can be used offline.
Encouraged by the existing real users of the project further improvements need to be done to meet the requirements of developers and cartographers using the project to create custom OSM maps without building up their own rendering pipeline. These actions will lead to further adoption and a healthy Open Source project that will survive this thesis.

%---------------------------------------

\section{Problems}\label{goals}

The focus off this bachelor thesis lies on three major problems that need to be solved.


\subsection{Scalable rendering process}\label{scalable-rendering-process}

The global vector tiles should be rendered within a reasonable time-frame
to meet project deadlines and enable developers to iterate quickly on the vector tiles.
Unfortunately to the sheer amount of tiles a single process on a single machine takes a half year to complete.
By distributing the process on multiple machines the time for rendering the planet can be significantly reduced only limited
by the amount of infrastructure available serving as example how to distribute a tile rendering pipeline and enabling
vector tiles with global coverage.

\subsection{Updatable vector tiles}

\osm{} contributors add several million changesets every day. Keeping a map up to date is of significant relevance to the users of the vector tiles and the contributors. The vector tiles should be released in a fixed weekly interval.
However rerendering the entire planet using the scalable rendering process is not feasible due to the infrastructure costs. By calculating the tiles that will change in advance and only render those tiles a single machine can keep the vector tiles up to date making the project sustainable for long term. 


\subsection{Cartographic Standards}

The vector tiles need to meet certain cartographic standards to enable cartographers to create high quality maps. Creating a global base map from scratch is a huge undertaking with several interesting problems like label placement, importance ranking and fitting the right data into less space. By focusing on quality and compatibility with Mapbox Streets v7 the vector tile data schema and makes the project truly usable for the end users which expect high quality maps.