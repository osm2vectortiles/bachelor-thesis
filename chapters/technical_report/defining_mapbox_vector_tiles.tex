\chapter{Defining Mapbox Vector Tiles}\label{chapter_defining_mapbox_vector_tiles}

The main requirement regarding the content of the vector tiles is to be compatible with the vector tiles of Mapbox. This allows people to seamlessly switch to \osmvt{} and use the same visual styles created with Mapbox Studio. 

\section{Approach}
Mapbox's tileset is called Mapbox Streets. Mapbox provides detailed documentation on what data is included in the Mapbox Streets vector tiles. The documentation contains a layer reference which defines the attributes a layer can have. However the zoom levels at which data is shown is not documented publicly as well as the \osm{} tags and constraints describing the data. To be able to reverse engineer Mapbox Streets this information had to be retrieved by analyzing the official Mapbox Streets vector tiles at different zoom levels.\\
In an iterative and time consuming process the mapping and queries were continuously improved until the vector tile output matches the data from Mapbox Streets very closely.

\section{Implementation}
This section describes the main components which needed to be implemented in order to generate Mapbox Streets compatible vector tiles.

\begin{figure}[H]
\centering
\includegraphics[width=1.0\textwidth]{images/osm_to_vectortiles_detailed}
\caption{Simplified process from data import to vector tile rendering}
\end{figure}

\clearpage

\subsection{Import Mapping}\label{sub_import_mapping}

Imposm3\cite{4_github_2015} is used to import the \osm{} data into the PostgreSQL database. Imposm3 satisfies the two purposes of filtering and mapping the \osm{} data.

\paragraph{Filter data} The mapping allows to filter explicitly by tags and defines which data is imported. This is important since only a subset of all \osm{} data is included in the vector tiles and  therefore not all \osm{} data needs to be imported.

\paragraph{Mapping data} The mapping allows to map \osm{} key/value pairs to a certain database table creating a structured and organized schema from semi-structured data. It takes care of constructing actual geometries from the \osm{} data model (\autoref{openstreetmap_data_model}).

\vskip 0.2in

The example definition in listing \autoref{definition_of_mapping} maps \osm{} tags with the key building and any value associated with that key into the table \textbf{building\_polygon}. The \textbf{building\_polygon} table has the columns id, geometry, underground, timestamp and type. The \textbf{underground} column in the database originates from the normalized value associated with the \osm{} key \textbf{building:levels:underground}.\\ During the import process Imposm3 transformes the \osm{} nodes, ways and relations to one of the geometry types point, linestring or polygon. The mapping is one of the most important aspects of the project and maps more than 400 individual tags.

\begin{listing}[H]
\begin{yamlcode}
building_polygon:
  type: polygon
  fields:
  - name: id
    type: id
  - name: geometry
    type: geometry
  - name: underground
    key: building:levels:underground
    type: integer
  - name: timestamp
    type: pbf_timestamp
  mapping:
    building:
    - __any__
\end{yamlcode}
\caption{YAML definition of a single table in the import mapping}
\label{definition_of_mapping}
\end{listing}
\clearpage

\subsection{Zoom Level Views}\label{zoom_level_views}

After the import process all \osm{} data which is required for the vector tiles is stored in the database. Since only a subset of the data in the tables is shown on a given zoom level, SQL views for each zoom level and layer were created. The zoom level views filter the tables to the rows which are shown on a given zoom level.

\begin{figure}[H]
  \centering
  \includegraphics[width=\textwidth]{images/buildings_z13_z14}
  \caption{Difference of zoom level view 13 and 14 in buildings layer}
  \label{zoom_level_view_building_layer}
\end{figure}

\autoref{zoom_level_view_building_layer} shows on the left side the building layer on zoom level 13 and on the right side the same layer on zoom level 14. On zoom level 13 only a subset of the data shown on zoom level 14 is visible. This is a classic example of using zoom level views to provide more details on higher zoom levels.
\\
Listing \autoref{definition_of_zoom_level_view} shows the definition of the SQL views on both zoom levels. The \textbf{WHERE} clause in the query of the view \textbf{building\_z13} filters the rows to buildings which have an area greater than 1700. That is the reason why less buildings are shown on right side of \autoref{zoom_level_view_building_layer}.

\begin{listing}[H]
\begin{sqlcode}
CREATE OR REPLACE VIEW building_z13 AS
    SELECT id AS osm_id, underground, geometry
    FROM osm_building_polygon
    WHERE ST_Area(geometry) > 1700;

CREATE OR REPLACE VIEW building_z14 AS
    SELECT id AS osm_id, underground, geometry
    FROM osm_building_polygon;
\end{sqlcode}
\caption{Definition of zoom level views of building layer}
\label{definition_of_zoom_level_view}
\end{listing}

Additionally zoom level views help to decouple the database tables which hold the actual data and the definition of the layer. This is very helpful for example if new data is added to a layer, as only the import mapping and the zoom level views need to be modified.

\subsection{Layer Definition}

The source project contains the definition of the layers inside the vector tiles. The definition contains metadata to access the database and a query which returns the necessary data for this layer. The listing \autoref{definition_of_layer} shows the definition of the layer \textbf{building}. The query does not directly access the database table \textbf{building\_polygon}. Instead it queries the zoom level views \textbf{building\_z13} and \textbf{building\_z14}.

\begin{listing}[H]
\begin{yamlcode}
- id: aeroway
    Datasource: 
      type: postgis
      table: |-
        (
          SELECT osm_ids2mbid(osm_id, is_polygon(geometry)) AS osm_id,
                 geometry, building_is_underground(underground) AS underground 
          FROM (
            SELECT * FROM building_z13
            WHERE z(!scale_denominator!) = 13
            UNION ALL
            SELECT * FROM building_z14
            WHERE z(!scale_denominator!) = 14
          ) AS building WHERE geometry && !bbox!
        ) AS data
    properties: 
      "buffer-size": 4
\end{yamlcode}
\caption{Definition of layer aeroway in the vector tile source project}
\label{definition_of_layer}
\end{listing}

The layer definition serves as input to the vector tile renderer (Mapnik). The tile renderer will execute every layer query for each tile and replaces expressions like \textbf{!scale\_denominator!} (zoom level) and \textbf{!bbox!} (extent of the tile) with the values of the current tile. If the query in listing \autoref{definition_of_layer} is executed for a tile on zoom level 12 it won't return any data as the \textbf{WHERE} clause will not match in both cases.
Whereas if it is executed on a tile on zoom level 13 all data of the zoom level view building\_z13 will be included in the layer building.

\begin{figure}[H]
\centering
\includegraphics[width=\textwidth]{images/road_water_building}
\caption{Layer road, water, and building features on same map}
\end{figure}

\subsection{Relation between Database Tables, Zoom Level Views and Layers}

The \autoref{aero_db_schema} shows how the database tables, zoom level views and layers are related to each other. This architecture helps to structure the \osm{} data inside the database and opens the possibility to optimize single zoom levels individually. The arrow in \autoref{aero_db_schema} describes the data flow.

\begin{figure}[H]
\centering
\includegraphics[width=0.8\textwidth]{images/aero_database_schema}
\caption{Data flow between database tables, zoom level views and layer}
\label{aero_db_schema}
\end{figure}

\section{Problems and Optimizations}

During the development process of the map a number of problems were discovered and optimizations were implemented. This section explains the most interesting problems in detail.

\subsection{Avoid Expensive Transformations in Zoom Level Views}

The purpose of the zoom level views is to filter the data to only contain rows that are shown on a specific zoom level. Expensive calculations or transformations should be avoided in these views or made in a preprocessing step.\\

For example there are multiple label layers which transform a polygon geometry to a point geometry and since calculating the centroid from a polygon is expensive, transformations like these result in bad rendering performance.\\
The reason for this is that every time the layer query gets executed, all rows of the view get transformed even though only a subset of the data is needed for the rendered tile (\autoref{point_calculation_zoom_level_view}). Therefore selecting the right rows and only executing the transformation on the smallest possible subset results in much better performance (\autoref{point_calculation_layer_definition}).

\begin{figure}[H]
\centering
\includegraphics[width=0.7\textwidth]{images/transformation_layer_definition}
\caption{Point calculation in layer definition}
\label{point_calculation_layer_definition}
\end{figure}

\begin{figure}[H]
\centering
\includegraphics[width=0.7\textwidth]{images/transformation_zoom_level_views}
\caption{Point calculation in zoom level view}
\label{point_calculation_zoom_level_view}
\end{figure}

\subsection{Classification Functions}

\osm{} contains a complex taxonomy of tags mapping to real-world objects. The tags are structured into layers but cartographers need another level of abstraction to style groups of features. For example designers want to set the color of all parks to green instead of styling tags such as dog parks, gardens and playground individually.
Therefore closely related tags are grouped together and are stored in the \textbf{class} field. This process is called classification and is essentially a mapping of many tag values to a single value.
\\\\
These classifications cover more than 400 individual tags grouped into more than 100 groups and are implemented by SQL functions as in listing \autoref{definition_of_classification_helper_function}. These class definitions are written in a YAML based format from which the SQL classification functions are generated.

\begin{listing}[H]
\begin{plpgsqlcode}
CREATE OR REPLACE FUNCTION landuse_class(type VARCHAR) RETURNS VARCHAR
AS $$
BEGIN
    RETURN CASE
        WHEN type IN ('park', 'dog_park', 'garden', 'playground') THEN 'park'
        WHEN type IN ('school', 'college', 'university') THEN 'school'
        WHEN type IN ('cemetery', 'christian', 'jewish') THEN 'cemetery'
    END;
END;
$$ LANGUAGE plpgsql IMMUTABLE;
\end{plpgsqlcode}
\caption{Definition of classification helper function}
\label{definition_of_classification_helper_function}
\end{listing}

The listing \autoref{definition_of_classification_helper_function} shows the simplified class function of the layer landuse. It takes the type value as input and returns the correct class value. 

\subsection{OpenStreetMap ID Transformation}

The data model of \osm{} consists of nodes, ways and relations. Every object gets its own OSM id assigned. This id is not unique across object types. Therefore one can find three objects with the same id but with a different object type.\\
While this works perfectly fine for \osm{}, this represents a problem because during the import process these \osm{} objects get transformed to PostGIS geometries. Objects of different types can get transformed to the same PostGIS geometry and therefore their ids would collide.
\\
In order to prevent this, the ids need to be transformed according to \autoref{osm_id_transformation} to make OSM ids unique within vector tiles \cite{103_mapbox.com_2016}.

\begin{table}[H]
\centering

\begin{tabular}{lll}
\hline
OSM type & Geometry type                  & \osm{} ID transform \\ \hline
node     & point                          & id x 10          \\
way      & linestring                           & (id x 10) + 1    \\
way      & polygon + polygon label points & (id x 10) + 2    \\
relation & linestring                           & (id x 10) + 3    \\
relation & polygon + polygon label points & (id x 10) + 4    \\
\end{tabular}
\caption{OSM id transformation}
\label{osm_id_transformation}
\end{table}

\subsection{Place Label Rank Calculation}\label{place_label_rank_calc}

Ranks are important for determining at which zoom level which places should be displayed. The NaturalEarth database contains places with scaleranks assigned by humans and is the most important source for better quality labels (historic places might be much more important despite having a very small population). This dataset is merged with the imported \osm{} data.
They can also be used to limit density at lower zoom levels to decrease data density. Using the scalerank and other information such as place type and population the actual rank called \textbf{localrank} is calculated. An example of localrank calculation can be seen in figure \autoref{localrank_calculation} (localrank is denoted as number inside circle).

\subsubsection*{Algorithm}

\noindent\begin{minipage}[t]{0.48\linewidth}
    \vspace{0pt}
    \begin{enumerate}  
        \item Divide map into grid
        \item Group labels by tile index
        \item Sort labels by scalerank, type and population within group
        \begin{enumerate}
            \item By scalerank ascending
            \item By type city, town, village, hamlet, suburb, neighbourhood
            \item By population descending
        \end{enumerate}
        \item Use the row number of the sorted labels as \textbf{localrank} for each feature
    \end{enumerate}
\end{minipage}
\hfill
\begin{minipage}[t]{0.48\linewidth}
    \vspace{-10pt}
    \begin{figure}[H]
    \centering
    \includegraphics[width=\textwidth]{images/rank_calculation.png}
    \caption{Localrank calculation}
    \label{localrank_calculation}
    \end{figure}
\end{minipage}