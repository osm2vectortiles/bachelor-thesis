\chapter{Project Management}\label{project-management}

\section{Software Development Process}


An agile approach based on SCRUM has been used as the process model of this project.
The maintainers of this project act as joint product owners while the backlog is managed not only
by the product owners but also by the community which adds features to the wish list.\\

At the beginning of each sprint features of the back log are estimated and scheduled for the
next release. After each sprint (two weeks) a new stable version is released. \\

In comparison to SCRUM there are no sprint reviews and retroperspective but regular meetings with
the thesis advisor to inform about project status are conducted. Regular meetings with the technical advisor help to solve problems together.

\subsection{GitHub}\label{github}
GitHub was used for planning and tracking of the tasks and milestones.
To provide a SCRUM board and burndown chart the ZenHub browser plugin has been used.


It has a big advantage over other project management tools, as the revision control and the issue tracking is at the same place.
Non project members can understand the thoughts behind certain decisions and communicate their ideas directly to team members which is important for an Open Source project.\\

An organization named osm2vectortiles has been created with the following repositories:

\begin{itemize}
\item
  \textbf{osm2vectortiles} contains the project
\item
  \textbf{imposm3} Custom fork of imposm3 to support timestamp field
\end{itemize}

\section{Schedule}

Because the project members where already familiar with the technologies and field of work
no elaboration phase or sprint was needed. After a long KickOff meeting the regular two week
sprints started.

\section{Sprints}

Sprints

Each milestones marks a special release version of the vector tiles.

\begin{table}[H]
\centering
    \begin{tabular}{p{2.5cm} p{10cm}}
    v1.1 &  Evaluate how to implement OSM diff updates\\ \hline
    v1.2 &  \\ \hline
    v1.3 &  \\ \hline
    v1.4 &  \\ \hline
    v1.5 &  \\ \hline
    v1.6 &  \\ \hline
    \end{tabular}
    \caption[Milestones]{Project sprints}
\end{table}

\section{Roles and Responsibilities}\label{roles-and-responsibilities}

\begin{table}[H]
\centering
    \begin{tabular}{p{3cm} p{9.5cm}}
Prof Stefan Keller & Thesis advisor responsible for supervising work and
assess the thesis.\\ \hline
Dr Petr Pridal &
Technical partner responsible for providing infrastructure and guidance in technical and map related questions.\\ \hline
Manuel Roth &
Maintainer\\ \hline
Lukas Martinelli &
Maintainer\\ \hline
    \end{tabular}
    \caption{Thesis contributors and their roles}
\end{table}

\section{Risks}\label{risks}

In contrast to the preceding study thesis the bachelor thesis is less risky due
to the increased knowledge of the field.

\begin{table}[H]
\centering
    \begin{tabular}{p{5.5cm} p{6.5cm} p{1.8cm}}
    \hline
    Risk & Measurement & Probability (1-6)\\
    \hline
    Infrastructure not sufficient & Switch to non school infrastructure and rely on external sponsors & 3\\
    Rendering updates takes too long & Early measurements & 3\\
    Unwanted Features & Open roadmap and feedback of community & 3\\
    Quality not sufficient & Invest sufficient time in perfecting the quality & 3\\
    \end{tabular}
    \caption{Risks and measurements}
\end{table}

%---------------------------------------------------------------
\newpage
\chapter{Quality Measures}\label{quality-measures}

\section{Testing}\label{testing}

The osm2vectortiles ecosystem is quite diverse with a big collection of small tools that all work together which makes testing really complex.

\section{Integration Test}

In Travis CI\cite{pm_5_travis-ci.org_2015}  the entire workflow was completed for a small data sample on each commit.
Because the entire workflow is configured with Docker Compose \cite{pm_6_docs.docker.com_2015} the CI server had to execute all import steps in serial order. This is a straightforward way to check if all components work together correctly
and although it is a simple setup it has helped tremendously during project development, catching bugs
like missing tables or SQL typos.

\begin{yamlcode}
script:
  - docker-compose up -d postgis
  - docker-compose up -d pgbouncer
  - docker-compose run import-osm
  - docker-compose run import-natural-earth
  - docker-compose run import-water
  - docker-compose run import-labels
  - docker-compose run import-sql
  - docker-compose run update-scaleranks
  - docker-compose run export-local
  - docker-compose up -d serve
  - curl "http://localhost:8080/index.json"
\end{yamlcode}

\section{Guidelines}\label{guidelines}
To have homogeneous software the contributors have settled on common guidelines in the beginning of the project.

\subsection{Releases}
Semantic versioning \cite{pm_7_preston-werner_2015} should be used for releases.
At the end of each milestone a new release will be created.
\newpage

\subsection{Git}\label{git}
\paragraph{Commit Messages}
The seven rules of great git commit
messages\cite{pm_8_chris.beams.io_2015} should be used.

\paragraph{Rewriting}
Git history should be kept clean and therefore local branches should be
squashed meaningfully.

\paragraph{Pulling}
To avoid unnecessary merge messages one should always use the
\texttt{-\/-rebase} parameter.

\subsection{Workflow}\label{git-workflow}
The Feature Branch Workflow\cite{pm_9_atlassian_git_tutorial_2015} should be used. Every project member has a local repository with a copy of the remote
repository. For each feature ticket in GitHub a separate branch
will be created. Once a ticket has been completed a pull request will be
created and needs to be merged into the \texttt{master} branch by an other 

\subsection{Coding Standards}

\paragraph{Bash} Bash was used for the Docker image entrypoints and follow
the rules of Defensive Bash Programming \cite{pm_10_lavi_2012}.

\paragraph{Python} Python code should stay PEP-8\cite{pm_11_python.org_2015} compliant and write idiomatic Python code according to PEP-20\cite{pm_12_python.org_2015}.

\paragraph{JavaScript} The JavaScript code is checked using ESLint\cite{pm_13_eslint.org_2015}

\paragraph{SQL} The PostgreSQL code is using upper case for the key words. Apart from nice formatted SQL code and functions should be used
to keep the queries DRY\cite{pm_14_wikipedia_2015}.

\paragraph{Dockerfile} Dockerfiles follow the best practices\cite{pm_15_docs.docker.com_2015} defined by Docker.

%---------------------------------------------------------------