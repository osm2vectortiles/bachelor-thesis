\chapter{Project Management}\label{project-management}

\section{Software Development Process}


An agile approach based on SCRUM has been used as the process model of this project.
The maintainers of the project act as joint product owners while the backlog is managed not only
by the product owners but also by the community which add features to the wish list and even contribute code to the repository.\\
At the beginning of each sprint features of the backlog are estimated and scheduled for the next release. After each sprint (two to three weeks) a new stable version is released.\\
In comparison to SCRUM there are no sprint reviews and retroperspective but regular meetings with the thesis advisor to inform about project status are conducted.
Regular meetings with the technical advisor help to solve problems together.

\subsection*{GitHub}\label{github}
GitHub was used for planning and tracking of the tasks and milestones.
To provide a SCRUM board and burndown chart the ZenHub browser plugin has been used.


It has a big advantage over other project management tools, as the revision control and the issue tracking is at the same place.
Non project members can understand the thoughts behind certain decisions and communicate their ideas directly to team members which is important for an Open Source project.\\

An organization named osm2vectortiles has been created with the following repositories:

\begin{itemize}
\item
  \textbf{osm2vectortiles} contains the project
\item
  \textbf{imposm3} Custom fork of imposm3 to support timestamp field
\end{itemize}

\section{Schedule}

Because the \osmvt{} members where already familiar with the technologies and field of work no elaboration phase was needed. Each sprint was tightly coupled to the next release.

\section{Milestones}

Each milestones marks a special release version of the vector tiles.

\begin{table}[H]
\centering
    \begin{tabular}{p{1.5cm} p{1.5cm} p{1.5cm} p{1.5cm} p{1.5cm}}
    Version & Date  & Resolved Issues & Merged PRs & Fixed Bugs\\
    \hline
    v1.1   & Mar 4  & 9 & 3 & 1 \\
    v1.2   & Mar 21 & 9 & 5 & 1 \\
    v1.3   & Apr 1  & 6 & 6 & 1 \\
    v1.4   & Apr 12 & 14 & 11 & 6 \\
    v1.4.1 & Apr 22 & 10 & 11 & 10 \\
    v1.4.2 & Apr 25 & 5 & 4 & 11 \\
    v1.5   & Apr 28 & 5 & 8 & 3 \\
    v2.0   & May 24 & 12 & 8 & 4 \\
    \end{tabular}
    \caption[Milestones]{Project sprints and statistics}
\end{table}

\section{Roles and Responsibilities}\label{roles-and-responsibilities}

\begin{table}[H]
\centering
    \begin{tabular}{p{3cm} p{9.5cm}}
Prof Stefan Keller & Thesis advisor responsible for supervising work and
assess the thesis.\\ \hline
Dr Petr Pridal &
Technical partner responsible for providing infrastructure \\ \hline
Manuel Roth &
Maintainer\\ \hline
Lukas Martinelli &
Maintainer\\ \hline
    \end{tabular}
    \caption{Thesis contributors and their roles}
\end{table}

\section{Risks}\label{risks}

In contrast to the preceding study thesis the bachelor thesis is less risky due
to the increased knowledge of the field.

\begin{table}[H]
\centering
    \begin{tabular}{p{4.5cm} p{7.5cm} p{1.8cm}}
    \hline
    Risk & Measurement & Probability (1-6)\\
    \hline
    Too slow update process& Early measurements of solution & 4\\
    Vector tiles oversized & Continuously measure vector tiles & 4\\
    Infrastructure not sufficient & Switch to non school infrastructure and rely on external sponsors & 3\\
    Unwanted Features & Open roadmap and feedback of community & 2\\
    Lacking quality & Regularly control whether defined quality measurements were complied with & 2\\
    \end{tabular}
    \caption{Risks and measurements}
\end{table}
