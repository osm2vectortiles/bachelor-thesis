\chapter{Requirements Specification}\label{requirements_specification}

This chapter describes the requirements for the project.

\section{Use Cases}\label{use_cases}

There are two main use cases for this project with use case described in \autoref{use_case_use_vector_tiles} being the most relevant one while use case described in \autoref{use_case_render_vector_tiles} is for a much narrower audience.

\subsection{Render Vector Tiles}\label{use_case_render_vector_tiles}

The user renders her/his own set of vector tiles. Since not all data of \osm{} is inside our vector tiles, some users may want to add or remove additional data. Assuming users that are interested in creating their own vector tiles clone the \osmvt{} repository these are potentially 70 users each month.

\subsection{Use Vector Tiles}\label{use_case_use_vector_tiles}

The user makes use of the prerendered vector tiles and wants to create her/his own basemap. Assuming users that are interested in using their own vector tiles read the documentation these are potentially 600 users each month.

\section{User Characteristics}\label{user_characteristics}

\subsection{Limited or no access to the internet}

Users which have the constraint of limited or no access to the internet can download vector tiles for the entire planet and serve their custom basemap locally.

\subsection{Can't rely on a third-party service}

Many organizations can't afford to rely on a third-party service and want to run their map on-premise.

\subsection{Customizing the look of the basemap}

In many use cases it is desirably to adjust the basemap to better match the design of a product.

\section{Requirements}\label{requirements}

The bachelor thesis consists out of three major requirements out of the use cases described in \autoref{use_cases}. The technically interesting problems of these requirements are described in \autoref{technical_report}.

\begin{itemize}
    \item Prerendered vector tiles from \osm{} for the entire planet (\autoref{chapter_scalable_rendering_process})
    \item Update functionality to keep up with future \osm{} changes (\autoref{chapter_updatable_vector_tiles})
    \item Vector Tiles compatible with Mapbox Streets v7 (\autoref{chapter_defining_mapbox_vector_tiles})
\end{itemize}


\section{Non Functional Requirements}\label{non_functional_requirements}

The non functional requirements are the key to success of this project. If the following requirements can be fulfilled, the specified users will be able to benefit form our project.

\paragraph{Performance}

The initial rendering process for the entire world must be kept below two weeks. The import on the master server should take less than a day while rendering should be kept below two weeks.
Updates should happen in a weekly interval.

\paragraph{Learnability}

It is important that users without previous vector tiles or Docker experience can get started with as few obstacles as possible.

\paragraph{Cost}

The cloud instances to render the world once costs around 1500 dollar. The updating should only cost a fraction of this initial investment.

\paragraph{Repeatability}

Since the \osmvt{} will provide continuous updates it is important that the process and results are repeatable.

\paragraph{Compatibility}

Compatibility with Mapbox Streets gives the users access
to a wide range of styles and editors. Therefore the vector tiles must contain
all features sets Mapbox Streets contains.

\paragraph{Vector Tile Size}

The size of a single vector tile should not be greater than 500 KB.
